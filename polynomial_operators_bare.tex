%\documentclass{homework} % comment to compile bare
\documentclass{amsart} % uncomment to compile bare
\usepackage{amsthm}
\usepackage{cleveref}
\usepackage{enumerate}
\newtheorem{lemma}{Lemma}
\newtheorem{thm}{Theorem}
\newtheorem{ex}{Example}

\title{Kevin Joyce}
%\course{Polynomial Differential Operators} % comment to compile bare
%\docdate{\today} % comment to compile bare
\author{Kevin Joyce}
\begin{document} 
\newcommand{\eps}{\varepsilon}
\newcommand{\del}{\partial}
\newcommand{\CC}{\ensuremath{\mathcal C}} % uncomment to compile bare
\newcommand{\RR}{\ensuremath{\mathcal R}} % uncomment to compile bare

\section{The characteristic polynomial} 

Recall ODE's with linear constant coefficients of the form
\begin{equation}
  y^{(n)} + a_1 y^{(n-1)} + \dots + a_0y = f(t) \label{diffeq}
\end{equation}
have unique solutions  
\begin{equation}
  y(t) = y_h(t) + y_p(t)
\end{equation}
where $y_h(t)$ solves the associated \emph{homogeneous} equation
\begin{equation}
  y^{(n)} + a_1 y^{(n-1)} + \dots + a_0y = 0 \label{homo}
\end{equation}
and $y_p(t)$ is a particular solution to \eqref{diffeq}.  We have classified the solutions to the homogeneous equation by writing \eqref{homo} in the form $AY = Y'$ where 
$$
  Y = \begin{pmatrix}
    y\\
    y'\\
    \vdots \\
    y^{(n-1)}
  \end{pmatrix}
  \quad\text{ and }\quad
  A = \begin{pmatrix}
    0 & 1 & 0 & \cdots & 0 \\
    0 & 0 & 1 & \cdots & 0 \\
    \vdots & \vdots & \cdots & \ddots & \vdots \\
    0 & 0 & \cdots & 0 & 1\\
    -a_0 & -a_1 & \cdots & -a_{n-2} & -a_{n-1}
  \end{pmatrix},
$$
and exponentiating $A$ for a given set of $n$ initial conditions; i.e. 
\begin{equation}
  Y = e^{At} Y_0 = P e^{\Lambda t} P^{-1} Y_0 \label{solution}
\end{equation} 
where $Y_0$ is an $n\times 1$ vector of arbitrary initial values and $A = P\Lambda P^{-1}$ is the Jordan decomposition of $A$.  Note that any eigenvalue satisfies $\det (A - \lambda I) = 0$. This determinant is relatively easy to compute using the Laplace expansion about the bottom row and the fact that the determinant of a lower triangular matrix is the product of the diagonal elements, so
\begin{align}
  &\det\begin{pmatrix}
    -\lambda & 1 & 0 & \cdots & 0 \\
    0 & -\lambda & 1 & \cdots & 0 \\
    \vdots & \vdots & \cdots & \ddots & \vdots \\
    0 & 0 & \cdots & -\lambda & 1\\
    -a_0 & -a_1 & \cdots & -a_{n-2} & (-a_{n-1}-\lambda)
  \end{pmatrix} = 0 \nonumber\\
  \iff&
  -a_0 -a_1\lambda - \dots - a_{n-2}\lambda^{n-2} - (a_{n-1} -\lambda)\lambda^{n-1} = 0 \nonumber\\
  \iff&p(\lambda) := \lambda^n + a_{n-1}\lambda^{n-1} + \dots + a_1 \lambda + a_0 = 0. \label{charpoly}
\end{align}

The polynomial defined in equation \eqref{charpoly} is called the
\emph{characteristic polynomial}.  Note that this is precisely the polynomial
one would get if they replaced $\lambda$ with $y$ in the left hand side of
$\eqref{homo}$ and substituted successive differentiation with multiplication.
This is generally how it is presented in a first ODE's course, and we see that the condition that it be zero
 coincides the linear algebara notioin of a characteristic polynomial for finding eigenvalues.   

\section{Polynomial operators and complex exponential functions}

We have shown that the characteristic polynomial, which in some sense determines
the homogeneous solutions to \eqref{homo}, coincides with the polynomial
operator which defines left hand side of \eqref{diffeq}.  We also know that the
matrix exponential of $\Lambda t$ in the Jordan decomposition results in
functions of the form $t^k e^{\lambda t}$, for $0\le k < n$, hence,
\eqref{solution} implies that all such solutions are linear combinations of
these.  It is natural consider this class of functions as ``right-hand-side''s
of \eqref{diffeq}, since often these are also defined as solutions to some other constant coefficient ODE
(think springs connected to other springs or RLC circuits in series).  It turns out
that these functions have particularly nice particular solutions.

Let us first establish some basic facts about polynomial operators with
constant coefficients. Let $D$ denote the differential operator, then for any
given polynomials $p$ and $q$, any sufficiently differentiable functions $f$
and $g$, and any constant $c$, the following propositions are relatively easy to show 
\begin{enumerate}[(i)] 
\item $p(D)(f + g) = p(D)f + p(D)g$
\item $p(D)(q(D)f) = (p(D)q(D)) f = (q(D)p(D)) f$
\item $(p(D) + q(D))f = p(D)f + q(D)f$
\item $p(D)(cf) = c\,p(D) f$.
\end{enumerate}

Using these facts, we can show that for exponential functions, polynomial operators act in a particularly nice way.
\begin{lemma}
  For any polynomial differential operator $p(D)$ and any complex exponential function $e^{\alpha t}$ with $\alpha \in \CC$ and $t \in \RR$, 
  $$
  p(D) e^{\alpha t} = p(\alpha) e^{\alpha t}.
  $$
\end{lemma}
\begin{proof}
  We need only expand $p(D)$, distribute, and compute,
  $$
    p(D) = (D^n + a_{n-1}D^{n-1} + \dots + Da_1 + a_0) e^{\alpha t} = \alpha^n e^{\alpha t} + a_n \alpha^{n-1} e^{\alpha t} + \dots + a_1 \alpha e^{\alpha t} + a_0 e^{\alpha t} = p(\alpha) e^{\alpha t}.
  $$
\end{proof}

As an immediate consequence of this fact, we can find particular solutions for \eqref{diffeq} where $f(t) = e^{\alpha t}$ and $p(\alpha) \not=0$.  We state the result in the following theorem and provide a few examples.

\begin{thm}
  For the constant coefficient linear ODE of the form $p(D) y = e^{\alpha t}$, where $p(\alpha) \not=0$, we have $y(t) = \frac{1}{p(\alpha)} e^{\alpha t}$ as a particular solution.
\end{thm}
\begin{proof}
  Using the properties of polynomial operators and the lemma, $p(D) \frac{1}{p(\alpha)} e^{\alpha t} = \frac{1}{p(\alpha)} p(D) e^{\alpha t} = 1 \, e^{\alpha t}$.
\end{proof}

\begin{ex} 
  Consider the example given in \emph{D'Angelo Example 1.2.} - Find a particular solution for $(D-5)(D-3)y = e^t$
\end{ex}
{\bf Solution:}
  We directly appeal to the theorem to obtain $y(t) = \frac{1}{(1-5)(1-3)} e^{t} = \frac 18 e^t$. Note that this agrees with the solution arrived at in \cite{dangelo} (pg. 14). 

\begin{ex}
  Find a particular solution for $(D - 2) y = \cos(3t)$.
\end{ex}
{\bf Solution:}
We cannot directly apply the theorem yet.  One could proceed by writing
$\cos(3t) = \frac 12 (e^{3ti} + e^{-3ti})$ and use the principle of
superposition, however, we will proceed by solving a related problem and translating it back to the original problem.  This ``meta-technique'' is a powerful idea that is used quite often in mathematics.

Let $z = y+ix$, and consider the complex ODE $(D - 2)z = e^{3ti}$. Note that
the real part of $e^{3ti} = \cos(3t)$.  By the theorem, the solution to this
ODE is given by $z = \frac 1{3i - 2} e^{3ti}$.  Now, we invoke the fact that
complex numbers are equivalent if and only if their real and imaginary parts
are equivalent; i.e. $y = \mathrm{Re}\, z = \mathrm{Re}\, \frac 1{3i - 2}
e^{3ti} = \mathrm{Re}\, \frac 1{13} (-2 - 3i) (\cos 3t + i\sin 3t) = \frac
1{13}(-2\cos 3t + 3\sin 3t)$.  Compare this with the method of undetermined
coefficients for $y = c_1 \cos(3t) + c_2 \sin(3t)$, and you'll see that this method
nicely consolidates the algebra using complex numbers.

It remains to find a particular solution to $p(D)y = e^{\alpha t}$ when $p(\alpha) = 0$.  We will need the following lemma:
\begin{lemma}
  Suppose $f$ is a sufficiently differentiable function, $p(D)$ is a polynomial
  differential operator, and $e^{\alpha t}$ is a complex exponential function
  with $\alpha \in \CC$ and $t \in \RR$.  Then the following identity holds,
  $$
    p(D) e^{\alpha t} f = e^{\alpha t} p(D + \alpha) f.
  $$
\end{lemma}
\begin{proof}
First, note that $D^k e^{\alpha t} f = D^{k-1} (e^{\alpha} D f + \alpha e^{\alpha t}f) = D^{k-1}e^{\alpha t} (D - \alpha) f$.  Proceeding inductively, we have $e^{\alpha t} (D-\alpha)^k f$.  We now apply this identity to each term in $p(D)e^{\alpha t} f$ to arrive at the desired identity.
\end{proof}

Now, let us consider the ODE $p(D) y = e^{\lambda t}$ where $p(\lambda) = 0$. Let us write $p(D) = (D-\lambda)^kq(D)$ where $k$ is the first integer such that $q(\lambda) \not=0$.  Note $(D-\lambda) e^{\lambda t} = 0$, hence we can form a \emph{new homogeneous} problem
\begin{equation}
  (D-\lambda)p(D) y = 0 \iff (D-\lambda)^{k+1}q(D) y = 0.
\end{equation}
Using similar reasoning to \eqref{solution}, we have that homogeneous solutions are of the form
\begin{equation}
  \tilde y = c_0e^{\lambda t} + c_1t e^{\lambda t} + \dots + c_kt^k e^{\lambda t} + \sum q_{j}(t) \label{newhomo}
\end{equation}
where $q_j(t)=c_jt^je^{\lambda_j t}$ are solutions corresponding to the factorization of the polynomial $q$. Now, if we allow $p(D)$ to operate on $\tilde y$, 
\begin{align*}
  p(D) \tilde y 
    &= (D-\lambda)^k q(D) \tilde y \\
    &= 0 + q(D) (D-\lambda)^k c_k e^{\lambda t} t^k + 0 \\
    &= c_k q(D) e^{\lambda t} (D-\lambda + \lambda)^k t^k \\
    &= c_k q(D) e^{\lambda t} k! \\
    &= c_k q(\lambda) k!e^{\lambda t}.\\
\end{align*}
So if $c_k = \frac 1{q(\lambda)k!}$ and $c_j = 0$ otherwise, then $\tilde y = \frac{t^k e^{\alpha t}}{q(\lambda)k!}$ is a particular solution for $p(D)y = e^{\lambda t}$. We remark that a similar technique can be used to motivate the method of undetermined coefficients.  See \cite{mit}.

For this to be useful, we need to be able to factor $p(D) = (D-\lambda)^k q(D)$
to obtain a particular solution. Note that if $\lambda$ is a root of order 1,
which is to say $p(z) = (z-\lambda)q(z)$ where $q(\lambda) \not=0$, then
$\left.\frac d{dz} p(z)\right|_{z=\lambda} = \left(q(z) +
(z-\lambda)\frac{d}{dz} q(z)\middle)\right|_{z=\lambda} = q(\lambda)$. A nice
exercise in advanced calcululs shows that $q$ is given by the first integer $k$
such that $\left.\left[\frac d{dz}\right]^kp(z)\right|_{z=\lambda} \not=0$.  We state this
result in the following theorem:
\begin{thm}
  For the constant coefficient linear ODE of the form $p(D) y = e^{\lambda t}$, where $p^{(j)}(\lambda)=0$ for $0\le j <k$ and $p^{(k)}(\lambda) \not= 0$ (i.e. $\lambda$ is a root of order $k$ for $p$), we have $$y(t) = \frac{t^k e^{\alpha t}}{k!p^{(k)}(\alpha)}$$ as a particular solution.
\end{thm}

Consider the following example
\begin{ex}
  Solve $(D^2 + 9)y = \cos 3t$ with $y(0) = 1$ and $y'(0) = 0$.
\end{ex}
{\bf Solution:}
Note that the corresponding complex ODE is $(D^2 + 9)z = e^{3it}$ has $p(3i) = -9 + 9 =0$.  Note also $p'(3i) = 2\cdot 3i + 0 = 6i \not= 0$.  The theorem gives a particular solution as $z(t) = \frac{ te^{3it} }{ 6\i }$.  Equating real parts, we have a particular solution $y_p(t) = \frac t6 \sin 3t$.  The homogeneous solutions are given by $c_1 \cos 3t + c_2 \sin 3t$, so the general solution is
$$
  y(t) = c_1 \cos 3t + \left(c_2 + \frac t6\right) \sin 3t.
$$
Applying the initial conditions, we have $c_1 = 1$ and $c_2 = 0$, and we arrive at
$$
  y(t) = \cos 3t + \frac t6 \sin 3t.
$$

We remark that as $t\to \infty$, $|y(t)|$ is arbitarily large for certain values of $t' > t$, despite the forcing function $f(t) = \cos 3t$ begin bounded. This phenomenon is sometimes called \emph{resonance}. Moreover, the freqeuncies of forcing functions where resonance occurs depends completely upon the characteristic equation of the defined system; i.e. they are precicely the roots of $p(\lambda)$.  In fact, in view of Theorem 1, we see that for forcing functions $e^{\alpha t}$ with $\alpha \approx \lambda$ will have large amplitudes in the particular solution, since near roots of $p$, $\left| \frac 1{p(\alpha)} \right|$ is very large.  So, there is, in some sense, a continous transition from large amplitude solutions to unbounded ones as $\alpha$ approaches roots $\lambda$.  See the Beats lecture in \cite{mit} for more information.

\begin{thebibliography}{1}
%    \bibitem{hinkley} D.~V.~Hinkley (1969). ``On the Ratio of Two Correlated Normal Random Variables.'' \emph{Biometrika}, {\bf 56}, pages 635-639. 
%    \bibitem{marsaglia} G.~Marsaglia (1965). ``Ratios of Normal Variables and Ratios of Sums of Uniform Varliables.''  \emph{Journal of the American Statistical Association}, {\bf 60}, pages 193-204.
    \bibitem{dangelo} D'Angelo, J. P. (2013). Hermitian Analysis. AMC, 10, 12.
    \bibitem{mit} Miller, H. and Mattuck, A. \emph{18.03 Differential Equations}, Spring 2010. (MIT OpenCourseWare: Massachusetts Institute of Technology), \\http://ocw.mit.edu/courses/mathematics/18-03-differential-equations-spring-2010\\ (Accessed 18 Mar, 2014). License: Creative Commons BY-NC-SA
    \end{thebibliography}
\end{document}

