\documentclass{homework}
\usepackage{cancel}
\usepackage{amsthm}
\usepackage{cleveref}
\usepackage{upgreek}
\usepackage[framed]{mcode}
\usepackage{mathrsfs}
\usepackage{units}
\usepackage{pgf,tikz}
\usetikzlibrary{arrows}
\usetikzlibrary{matrix}
\newtheorem{lemma}{Lemma}

\title{Kevin Joyce}
\course{Math 564 - Hermetian Analysis - Homework 4}
\author{Kevin Joyce}
\docdate{\today}
\begin{document} 
\newcommand{\figref}[1]{\figurename~\ref{#1}}
\renewcommand{\bar}{\overline}
\renewcommand{\hat}{\widehat}
\renewcommand{\SS}{\mathcal S}
\newcommand{\eps}{\varepsilon}
\newcommand{\TTheta}{\overline{\underline \Theta} }
\newcommand{\del}{\partial}
\newcommand{\approxsim}{\overset{\cdotp}{\underset{\cdotp}{\sim}}}

\problem{{\bf D'Angelo 1.32.} Define a sequence of functions $\{f_n\}$ on $[0,1]$ as follows: $f_n(x) = 0$ for $9\le x\le \frac 1n$ and $f_n(x) = -\log(x)$ otherwise.}  

\begin{solution}
Note that $ \lim_{n\to\infty} f_n(x) = -\log x $ if $x > 0$ and $\lim_{n\to\infty} f_n(0) = 0$. The resulting function is unbounded, hence it is not Riemann integrable. However, for $m<n$,
$$
  \|f_n - f_m\|_{L_2}^2 = \int_0^1 (f_n(x) - f_m(x))^2 dx = \int_{1/n}^{1/m} \log^2(x)dx.
$$ 
By the mean value theorem,
\begin{align*}
\int_{1/n}^{1/m} \log^2(x)dx 
  &= \left(\frac 1m - \frac 1n\right) \log^2(c) \quad\text{ for some }c \in \left(\frac 1n,\frac 1m\right)\\
  &\le \left(\frac 1n \right) \log^2\left(\frac 1n\right).
\end{align*}
To evaluate this limit, we reparameterize and apply L'Hopital's rule twice
\begin{align*}
  \lim_{x\to0} y \log^2(y) 
  &= \lim_{x\to0} \frac{ 2x^{-1} \log x }{-x^{-2}} \\
  &= \lim_{x\to0} \frac{ -2\log x }{x^{-1}} \\
  &= \lim_{x\to0} \frac{ -2 x^{-1}}{-x^{-2}} \\
  &= \lim_{x\to0} 2x \\
  &= 0 \\
\end{align*}
Hence $\{f_n\}$ is Cauchy with respect to the $L_2$ norm, yet does not converge to a Riemann integrable function.  

If we take $m<n$ sufficiently small so that $|\log(1/m)| > 1$, then $|\log(x)| < |\log(x)|^2$ for $1/n<x<1/m$ and 
$$
  \|f_n - f_m\|_{L_1} = \int_0^1 |f_n(x) - f_m(x)| dx = \int_{1/n}^{1/m} |\log(x)|dx \le \int_{1/n}^{1/m} \log^2(x) dx = \|f_n - f_m\|_{L_2}.
$$
So the sequence is Cauchy in $L_1$ as well.

\end{solution}

\problem{{\bf D'Angelo 1.35.} For $0\le r < 1$, define $P_r(\theta)$ as follows.  Put $z= re^{i\theta}$ and put $P_r(\theta) = \frac{1 - |z|^2}{|1-z|^2}$.  Show 
$$
  P_r(\theta) = \sum_{n\in\ZZ} r^{|n|}e^{in\theta}.
$$ }

\begin{solution}
  We calculate
  \begin{align*}
  P_r(\theta) 
  &= \sum_{n\in\ZZ} r^{|n|}e^{in\theta} \\
  &= \sum_{n=0}^{\infty} r^{|n|}e^{in\theta} + \sum_{n=1}^{\infty} r^{|-n|}e^{-in\theta} \\
  &= \sum_{n=0}^{\infty} z^n + \sum_{n=1}^{\infty} \bar z^n \\
  &= \frac{1}{1 - z} + \frac{\bar z}{1 - \bar z} \\
  &= \frac{(1-\bar z) + \bar z (1-z)}{(1 - z)(1 - \bar z)} \\
  &= \frac{1-\bar z z}{(1 - z)\bar{(1 - z)}} \\
  &= \frac{1-|z|^2}{|1 - z|^2}. \\
  \end{align*}
\end{solution}

\newcommand{\GG}{\ensuremath{\mathcal G}}
\problem{{\bf D'Angelo 1.36.} For $0 < t < \infty$, put $\GG_t(x) = \sqrt{\frac t\pi} e^{-tx^2}$. Then $\GG_t$ defines an approximate idenity.}  

\begin{solution}
Since $\GG_t(x) > 0$, we need only to show that $\int_{-\infty}^\infty \GG_t(x)dx = 1$ and, that for $\delta > 0$,
$$
  \lim_{t\to\infty} \int_{|x|\ge \delta} \GG_t(x)dx = 0.
$$

First, we calculate
\begin{align*}
  \int_{-\infty}^\infty \GG_t(x) dx
  &= \pi^{-1/2} \int_{-\infty}^\infty e^{-tx^2} \sqrt{t}dx\\
  &=  \pi^{-1/2} \int_{\infty}^\infty e^{-u^2} dx\\
  &\stackrel *=  \pi^{-1/2} \sqrt{\pi}\\
  &= 1.
\end{align*}

The equality in $*$ can be seen by noting the convergence of $\int u^{-2} \ge \int e^{-u^2}$, and evaluating the square of the integral by polar coordinates.  I.e.

$$
  \left(\int_{-\infty}^{\infty}e^{-u^2}du\right)^2 = \int_0^{2\pi}\int_0^\infty e^{-r^2}rdrd\theta = \int_0^{2\pi}2d\theta = \pi.
$$

Now, since $\GG_t$ is even, and using the same change of variables as before
\begin{align*}
  \lim_{t\to\infty} \int_{|x|\ge \delta} \GG_t(x)dx 
  &= 2 \int_{-\infty}^{-\delta} \GG_t(x)dx\\
  &= \frac 2{\sqrt \pi} \int_{-\infty}^{-\delta \sqrt t} e^{-u^2}dx = 2 F(-\delta \sqrt t),\\
\end{align*}
where $F:(-\infty,0] \to (0,1/2]$, defined by 
$$
  F(y) \frac 1{\sqrt \pi} \int_{-\infty}^{y} e^{-u^2}dx.
$$
To see that $F$ does indeed map into $(0,1/2]$, recall that the integral over
the whole real line was shown to be $1$, and since the integrand is even, the
integral over $(-\infty,0]$ is $1/2$. Since $e^{-u^2} > 0$, $F$ is increasing,
and by the fundamental theorem of calculus, $F$ is continuous.  Moreover, $\lim_{y\to\infty} F(y) = 0$.  Hence, for a given $\eps > 0$there exists a $y^*$ so that $ F(y^*) = \eps/2$, by the
intermediate value theorem, .
Now, choose $t > \frac{y^*}{\delta}$ then $-\delta \sqrt t < y^*$ implies 
$$
 \int_{|x| \ge \delta} \GG_t(x) dx= 2 F(-\delta \sqrt t) < \eps.
$$
\end{solution}

\problem{{\bf D'Angelo 1.38.} Find the Fourier series for $\cos^{2N}(\theta)$.}

\begin{solution}
Using the complex identity for $\cos(\theta)$, we calculate
\begin{align*}
  \cos^{2N}(\theta) 
  &= \frac{1}{2^{2N}} (e^{i\theta} + e^{-i\theta})^{2N}\\
  &= \frac{1}{2^{2N}} \sum_{j=0}^{2N} e^{i\theta j}e^{-i\theta(2N-j)} \\
  &= \frac{1}{2^{2N}} \sum_{j=0}^{2N} e^{i\theta(2N-2j)} \\
  &= \frac{1}{2^{2N}} \left(\sum_{j=0}^{N} e^{2i\theta(N-j)} + \sum_{j=N+1}^{2N} e^{2i\theta(N-j)}\right)\\
  &\stackrel\dagger= \frac{1}{2^{2N}} \left(\sum_{k=0}^{N} e^{2i\theta k} + \sum_{k=1}^{N} e^{-2i\theta k}\right)\\
  &= \sum_{k=-N}^N c_k e^{i\theta k} \quad \text{ where } c_k = \frac 1{2^{2N}}\text{ if $k$ is even, $0$ otherwise.}
\end{align*}

Incidentally, we can easily calculate $\int \cos^{2N}$ from $\dagger$ by
\begin{align*}
\int \cos^{2N}(\theta)\, d\theta 
  &\stackrel\dagger= \frac{1}{2^{2N}} \left(\sum_{k=1}^{N} \int e^{2i\theta k} d\theta + \sum_{k=1}^{N} \int e^{-2i\theta k} d\theta + \theta\right) + c\\
  &= \frac{1}{2^{2N}} \left(\sum_{k=1}^{N} \frac 1{2ik} e^{2i\theta k} + \sum_{k=1}^{N} \frac {-1}{2ik} e^{-2i\theta k}  + \theta\right) + c\\
  &= \frac{1}{2^{2N}} \left(\sum_{k=1}^{N} \frac 1{2ik} \left(e^{2i\theta k} - e^{-2i\theta k}\right)  + \theta\right) + c\\
  &= \frac{1}{2^{2N}} \left(\sum_{k=1}^{N} \frac 1{k} \sin(2k\theta)  + \theta\right) + c.\\
\end{align*}
\end{solution}

\end{document}
