\documentclass{homework}
\usepackage{cancel}
\usepackage{amsthm}
\usepackage{cleveref}
\usepackage{upgreek}
\usepackage[framed]{mcode}
\usepackage{mathrsfs}
\usepackage{tikz}
\usepackage{units}
\usetikzlibrary{matrix}
\newtheorem{lemma}{Lemma}

\title{Kevin Joyce}
\course{Math 564 - Hermetian Analysis}
\author{Kevin Joyce}
\docdate{\today}
\begin{document} 
\newcommand{\figref}[1]{\figurename~\ref{#1}}
\renewcommand{\bar}{\overline}
\renewcommand{\hat}{\widehat}
\renewcommand{\SS}{\mathcal S}
\newcommand{\eps}{\varepsilon}
\newcommand{\TTheta}{\overline{\underline \Theta} }
\newcommand{\del}{\partial}
\newcommand{\approxsim}{\overset{\cdotp}{\underset{\cdotp}{\sim}}}

\problem{{\bf D'Angelo 1.4} Find a sequence of complex numbers such that $\sum a_n$ converges but $\sum(a_n)^3$ diverges. }
\begin{solution}
  Let $\omega = e^{2\pi i/3 }$ and note $\omega^3 = 1$. Now define
  $$
    a_n = \frac{\omega^n}{n^{1/3}},
  $$
  and note $\{n^{-1/3}\}$ is a decreasing sequence converging to 0. Hence 
  $$\sum|n^{-1/3} - (n+1)^{-1/3}| = \sum(n^{-1/3} - (n+1)^{-1/3}) = 1 - 0.$$ Also, 
  $$
    \left|\sum_{n=1}^N \omega^n\right| = \left|\omega\frac{1-\omega^N}{1-\omega}\right| \le \frac{2}{|1-\omega|.}
  $$
  By Corollary 1.1, $\sum a_n$ converges.  On the other hand, 
  $$
    \sum_{n=1}^N a_n^3 = \sum_{n=1}^N \frac{1}{n}
  $$
  is the diverging harmonic series.
\end{solution}

\problem{{\bf D'Angelo 1.5 Modified} The following statement is equivalent to that stated in the text.  Let $(X,\|\cdot\|)$ be a normed vector space. Prove $X$ is complete if and only if $\sum a_n$ converges whenever $\sum\|a_n\|$ converges.}
\begin{solution}
First suppose that $X$ is complete. Now, take $a_n\in X$ such that $\sum \|a_n\|$ converges. Hence $\sum\|a_n\|$ is a Cauchy sequence in $\RR$. We will show $\sum a_n$ is Cauchy. To that end, take $k>j$ and observe
$$
  \left|\sum_{n=j}^k a_n\right| \le \sum_{n=j}^k \|a_n\|\quad\text{ by the triangle inequality}.
$$
Since $\sum\|a_n\|$ is Cauchy, the desired quantity can be made arbitrarily small.

On the other hand, suppose $\sum a_n$ converges whenever $\sum \|a_n\|$. Let $\{b_n\}$ be a Cauchy sequence in $X$, and we will show that it converges to some limit in $X$.  First, define $n_1$ so that $|b_{n_1} - b_n| < 1/2$ whenever $n\ge n_1$ guaranteed by $\{b_n\}$ Cauchy. Now take $n_k> n_{k-1}$, so that $|b_{n_k} - b_n| < 2^{-k}$ whenever $n\ge {n_k}$. Observe
$$
  \sum_{k=1}^N|b_{n_k} - b_{n_{k+1}}| < \sum_{k=1}^N 2^{-k} < 1.
$$
So $\sum|b_{n_k} - b_{n_{k+1}}|$ converges. Now, by assumption,
$$
  \sum_{k=1}^N b_{n_k} - b_{n_{k+1}} = b_{n_1} - b_{n_{N+1}} \to b
$$
for some $b\in X$. Thus, $b_{n_k}$ converges to $\tilde b = b_{n_1} - b$.  Now, for any $\eps > 0$ given, take $n$ and $n_k$ sufficiently large so that
$$
  |b_n - \tilde b| \le |b_n - b_{n_k}| + |b_{n_k} - \tilde b| < \eps/2 + \eps/2 = \eps.
$$
\end{solution}
\newpage
\problem{{\bf D'Angelo 1.6.} For $x<x<2\pi$, show that $\sum_{n=0}^\infty \frac{\cos(nx)}{\log(n+2)}$ converges to a non-negative function. }
\begin{solution}
Denote 
$$
  S_N = \sum_{n=0}^N \frac{\cos(nx)}{\log(n+2)}= \sum_{n=0}^N a_n b_n,
$$
where $a_n = 1/\log(n+2)$ and $b_n = \cos(nx)$.  $\{S_N\}$ converges by a similar argument to exercise 1.2; i.e. $a_n$ is a decreasing sequence converging to 0 and $B_N = \sum b_n$ is bounded by the same constant that $\sum \sin(nx)$ is.  It remains to show that this limit is non-negative.

Summing $S_N$ by parts twice, we have
\begin{align*}
  S_N &= a_N B_N - \sum_{j=0}^N (a_{j+1} - a_j)B_j \\
  &= a_N B_N - \left( (a_{N+1} - a_N) \sum_{j=0}^NB_j - \sum_{j=0}^N (a_{j+2} - 2a_{j+1} + a_j) \sum_{k=0}^jB_k\right)\\
  &= a_NB_N + (a_N - a_{N+1}) \sum_{j=0}^NB_j + \sum_{j=0}^N (a_{j+2} - 2a_{j+1} + a_j) \sum_{k=0}^jB_j.\\
\end{align*}

Note that $a_N B_N \to 0$ as $N\to \infty$, since $B_N$ is bounded and $a_N\to 0$. If we show that $(a_{N+1} - a_N) \sum_{j=0}^N B_j \ge 0$ and that $(a_{j+2} - 2a_{j+1} + a_j)\sum_{k=0}^j B_j \ge 0$ for all $j$, then for any given $\eps >0$ and sufficiently large $N$
$$
  S_N = \eps + C_N,\quad\text{where }C_N = (a_N - a_{N+1}) \sum_{j=0}^N B_j + (a_{j+2} - 2a_{j+1} + a_j)\sum_{k=0}^j B_j.
$$
Hence, $\lim S_N = \lim C_N$, which consists of non-negative terms, and thus, $\lim S_N$ will be shown to be non-negative.
%We will show that the sequence $\{B_N(2a_N - a_{N+1})\}$ converges to zero and that the sequence
%\begin{equation}
%  C_N = \left\{ \sum_{j=0}^N (a_{j+2} - 2a_{j+1} + a_j) B_J \right\}
%\end{equation}
%consists of non-negative terms.  Having shown these claims, then $C_N$ converges since $\{S_N\}$ and $\{-B_N(2a_N - a_{N+1})\}$ converge. Moreover, $\lim S_N = \lim C_N$, and $\lim S_N \ge 0$ since $C_N \ge 0$ and the proof will be complete.
%

We first show
\begin{align*}
 \sum_{j=0}^N B_j 
 &= \sum_{j=0}^N\sum_{n=0}^j \cos(nx) \\
 &= \sum_{j=0}^N\sum_{n=0}^j \frac{\omega^n + \bar \omega^n}{2},\quad\text{ where }\omega = e^{ix} \\ 
 &= \sum_{j=0}^N \frac 12 \left(\frac{1 - \omega^{j+1}}{1 - \omega} + \frac{1-\bar\omega^{j+1}}{1-\bar\omega} \right)\\ 
 &= \Re\left[ \sum_{j=0}^N \frac{1 - \omega^{j+1}}{1 - \omega} \right]\\ 
 &= \frac{1}{|1-\omega|^2}\Re\left[ \sum_{j=0}^N 1 - \omega^{j+1} -\bar\omega + \omega^j  \right]\\ 
 &= \frac{1}{|1-\omega|^2}\Re\left[  N(1 -\bar\omega) + (1-\omega)\sum_{j=0}^N \omega^j\right]\\ 
 &= \frac{N(1-\cos x) + \Big[1-\cos\Big((N+1)x\Big)\Big]}{|1-\omega|^2} \\
 &\stackrel{\dagger}\ge 0.\\ 
\end{align*}

Now let $f(x) = 1/\log x$ for $x>1$ and note $a_N - a_{N+1} = f(N+2) - f(N+3)$. Observe for $x>1$,
\begin{align*}
  f(x) - f(x+1) 
  &= -\int_x^{x+1} f'(t)dt\\
  &= -\int_0^1 f'(u+x)du\\
\end{align*}
and 
$$
  f'(x) = \frac{-1}{x\log x}.
$$
Hence $0 \le f(N+2) - f(N+3) = a_N - a_{N+1}.$ This together with $\dagger$, we have that the first term in $C_N$ is non-negative.

Now, take $f(x)$ as before, and observe
\begin{align*}
  f(x) - 2f(x+1) + f(x+2)
  &= \big(f(x+2) - f(x+1)\big) - \big(f(x+1) - f(x)\big)\\
  &= \int_{x+1}^{x+2} f'(t)dt - \int_x^{x+1}f'(t)dt\\
  &= g(x+1) - g(x),\quad\quad\text{where }g(x) = \int_x^{x+1}f'(t)dt\\
  &= \int_x^{x+1} g'(t) dt \\
  &= \int_0^1 g'(u+x)du. 
\end{align*}
So it suffices to show that $g'(x) \ge 0$. Well,
\begin{align*}
  g'(x) &= \frac{d}{dx}\left[ \int_a^{x+1} f'(t)dt - \int_a^xf'(t)dt\right]\quad \text{for }a>1\\
  &= f'(x+1) - f'(x) \\
  %&= f''(c) \quad\text{for some } c\in(x,x+1)\text{ by the Mean Value Theorem}.
  &= \int_x^{x+1} f''(t)dt.
\end{align*}
Finally we calculate 
$$
  f''(c) = \frac{\log(c)(\log(c) + 2)}{\left(c\log(c)^2\right)^2} \ge 0.
$$
Again, this with $\dagger$, shows that the terms of the partial sum of the second term in in $C_N$ are non-negative.  We have shown $C_N$ to be non-negative, and thus, the assertion is proved.
\end{solution}

\problem{{\bf D'Angelo 1.7} Put $f(\theta) = 1 + a\cos(\theta)$.  Note that $f\ge0$ if and only if $|a| \le 1$.  In this case, find $p$ such that $|p(e^{i\theta})|^2 = |f(\theta)|$.}
\begin{solution}

Note that if $z$ is such that $|z| = 1$, then for $z = e^{i\theta}$, the trigonometric polynomial
$$
  \frac a2 z^{-1} + 1 + \frac a2 z = 1 + a\cos(\theta) = f(\theta).
$$
Take 
\begin{align*}
  q(z) 
  &= z\left(\frac a2 z^{-1} + 1 + \frac a2 z\right)\\
  &= \frac a2 \left(z^2 + \frac 2az + 1\right).
\end{align*}
The two roots of $q$ are given by 
$$
-\frac 1a \pm \sqrt{\frac{1}{a^2} - 1},
$$
and note that they are multiplicative inverses of each other.  Hence, denote them as
$$
\xi = -\frac 1a + \sqrt{\frac{1}{a^2} - 1},\quad \xi^{-1}=-\frac 1a - \sqrt{\frac{1}{a^2} - 1}.
$$
Writing $q$ in factored form, we have
\begin{align*}
  q(z) 
  &= \frac a2 (z - \xi)(z - \xi^{-1})\\
  &= \frac a2 (z - \xi) \left(\frac 1{\bar z} - \frac 1\xi\right)\\
  &= \frac a2 (z - \xi) \frac {-1}{\bar z \xi} (\bar z - \xi)\\
  &= \frac{-a}{\bar z \xi} |z- \xi|^2
\end{align*}
Therefore, if 
$$
  p(z) = \left|\frac{a}{2\xi}\right|^{1/2} \cdot (z - \xi),
$$
then $|p(z)|^2 = |q(z)| = |z|\, \left|\frac a2 z^{-1} + 1 + \frac a2 z\right| = |1 + a\cos\theta|$.
%$$
%  p(z) = \left|\frac{a}{2\xi}\right|^{1/2} \cdot (z - \xi),\quad\text{ where } \xi = -\frac 1a + \sqrt{\frac{1}{a^2} - 1}.
%$$
%Note that $\xi^{-1} = -\frac 1a - \sqrt{\frac{1}{a^2} - 1}$, and that $\xi + \xi^{-1} = -\frac 2a$.  For $|z| = 1$, we now calculate
%\begin{align*}
%  |p(z)|^2
%  &= \left|\frac{a}{2\xi}\right| |z-\xi|^2\\
%  &= \left|-\frac{a}{2\xi}\right| |(z - \xi)(\bar z - \xi)| \\
%  &= \left|\frac{a\bar z}{2}(z - \xi)(z- \xi^{-1})\right| \\
%  &= |\bar z|\left|\frac{a}{2}(z^2 -z(\xi + \xi^{-1}) + 1)\right| \\
%  &= \left|\frac{a}{2}z^2 +z + \frac{a}{2}\right| \\
%  &= |z|\left|\frac{a}{2}z +1 + \frac{a}{2}\bar z\right| \\
%  &= |1 + a \cos\theta| \\
%%  &= \frac{|a|}{2|\xi|} (z- \xi)(\bar z - \xi)\\
%%  &= \frac{|a|\xi\bar z}{2|\xi|} (z- \xi)\frac{(\bar z - \xi)}{\xi \bar z}\\
%%  &= \frac{|a|\xi\bar z}{2|\xi|} (z- \xi)\left(\frac1\xi - \frac1{\bar z}\right)\\
%%  &= \frac{|a|\xi\bar z}{2|\xi|} (z- \xi)\left(\xi^{-1} - z\right)\\
%%  &= \frac{|a|\xi\bar z}{2|\xi|} \left(\xi^{-1}z -1 -z^2 + \xi z\right)\\
%%  &= \frac{|a|\xi\bar z}{2|\xi|} \left(-z^2 + \frac 2a z - 1\right)\\
%%  &= \frac{\xi}{|\xi|} \left(-|a|\cos(\theta) + \frac {|a|}a \right)\\
%%  &= \frac{|a|}{2|\xi|} \left(|z| -\bar z \xi - |z| \xi z + \xi^2 |z|\right)\\
%%  &= \frac{|a|}{2|\xi|} \left(1 -\bar z \xi - \xi z + \xi^2 \right)\\
%\end{align*}
\end{solution}

\end{document} 
