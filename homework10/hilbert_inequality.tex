\documentclass{homework} % comment to compile bare
%\documentclass{amsart} % uncomment to compile bare
\usepackage{amsthm}
\usepackage{cleveref}
\usepackage{enumerate}
\newtheorem{lemma}{Lemma}
\newtheorem{thm}{Theorem}
\newtheorem{ex}{Example}
\DeclareMathOperator*{\argmin}{arg\,min}

\title{Kevin Joyce}
\course{Hilbert's Inequality and Polynomial Projection in $L^2$} % comment to compile bare
\docdate{\today} % comment to compile bare
\author{Kevin Joyce}
\begin{document} 
\newcommand{\eps}{\varepsilon}
\newcommand{\del}{\partial}

Hilbert's inequality states that for a sequence of absolutely square summable complex numbers, $\{z_n\}$ with $\sum |z_n|^2 < \infty$, then
\begin{equation}
  \left| \sum_{j,k=0}^\infty \frac{z_j\bar z_k}{1 + j + k}\right| \le \pi \sum_{k=0}^\infty |z_k|^2. \label{hilbert_ineq}
\end{equation}
Moreover, $\pi$ is the smallest possible constant in the above inequality \cite{dangelo}.  We note that if $C$ is an infinite matrix whose $j,k$ entry is given by $1/(1+j+k)$ for $j=0,1,\dots$ and $k=0,1,\dots$, then the left hand side of \eqref{hilbert_ineq} can be viewed as the quadratic form $z C \bar z$ where $z = (z_0,z_1,\dots)$.  A finite version of the matrix $C$ also appears when projecting elements of $L^2$, the Hilbert space of integrable functions with respect to the standard Hermetian integral inner product, onto the space spanned by monomials of the form $\{z^k\}$.  That is, it was shown that for the finite basis $\{z_0,z_1,\dots,z_n\}$ and $w \in L^2[0,1]$, the solution to the minimization problem
\begin{equation}
  \argmin_{\alpha_k\in \CC^n} \left\|w - \sum_{k=1}^n \alpha_k z_k\right\|^2 \label{projection}
\end{equation}
was equivalent to solving the matrix equation
\begin{equation}
  C^T \alpha = v_{z,w}
\end{equation}
where $w_{z,b}$ is a vector depending on inner products of $z^k$ and $w$.  

For my project, I hope to investigate the relationship between these two
problems further and hopefully formulate Hilbert's inequality as an appropriate
projection problem.  In this way, it may be possible to say something useful
about the sharpness of $\eqref{hilbert_ineq}$, which was not proved in
\cite{dangelo}. This investigation will proceed by attempting to accomplish the following goals: generalize the projection
statement \eqref{projection} to linear combinations of countably infinite sets and possibly to more general sets than $[0,1]$; Formulate
a version Hilbert's inequality in the context of the general projection; Use the projection estimates
to evaluate the sharpness of the inequality.

The results of this work will culminate in a short technical report and presentation to be given during Spring semester finals week.

\begin{thebibliography}{1}
%    \bibitem{hinkley} D.~V.~Hinkley (1969). ``On the Ratio of Two Correlated Normal Random Variables.'' \emph{Biometrika}, {\bf 56}, pages 635-639. 
%    \bibitem{marsaglia} G.~Marsaglia (1965). ``Ratios of Normal Variables and Ratios of Sums of Uniform Varliables.''  \emph{Journal of the American Statistical Association}, {\bf 60}, pages 193-204.
    \bibitem{dangelo} D'Angelo, J. P. (2013). Hermitian Analysis. AMC, 10, 12.
    \end{thebibliography}
\end{document}

